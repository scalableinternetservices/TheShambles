\chapter{Conclusion}
Recent days have seen an upsurge in the number of data-driven applications. With the prominence of these applications, scalability has posed itself as a major concern. In this project, we tried to demonstrate how we can make our application scalable. We proposed some methods for enhancing the scalability of our application. We justified the employment of these methods with the help of load-testing our application using Tsung Framework.

While load-testing our application, we tried to construct generic scenarios which we deemed to be common actions for most users. We followed an iterative approach to improvement. That is, we first test our application and try to find a bottleneck. Once we localte a bottleneck, we try to remediate it. Once it has been taken care of, we run the tests again. This process continues as long as it is feasible to achieve better performance in the next iteration over the previous one.

In particular, we used the following techniques for improving the performance of our web app: pagination, AJAX, indexing, query optimization, horizontal and vertical scaling, and caching. The sheer amount of seed data that we had alone created obstacles in reasonable performance improvement. However, we believe our data model closely resembles most modern websites that tend to be heavily data-driven. Although we have tested our web application extensively with Tsung, we have to keep in mind that Tsung is not capable of simulating many aspects of a client interacting with a browser. Apart from that, we are fairly confident that we were able to introduce necessary improvements for boosting the performance of our web application and support it through Tsung testing.
